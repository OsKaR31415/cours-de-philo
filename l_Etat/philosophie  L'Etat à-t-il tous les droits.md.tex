% Options for packages loaded elsewhere
\PassOptionsToPackage{unicode}{hyperref}
\PassOptionsToPackage{hyphens}{url}
%
\documentclass[
]{scrartcl}
\usepackage{lmodern}
\usepackage{amssymb,amsmath}
\usepackage{ifxetex,ifluatex}
\ifnum 0\ifxetex 1\fi\ifluatex 1\fi=0 % if pdftex
  \usepackage[T1]{fontenc}
  \usepackage[utf8]{inputenc}
  \usepackage{textcomp} % provide euro and other symbols
\else % if luatex or xetex
  \usepackage{unicode-math}
  \defaultfontfeatures{Scale=MatchLowercase}
  \defaultfontfeatures[\rmfamily]{Ligatures=TeX,Scale=1}
\fi
% Use upquote if available, for straight quotes in verbatim environments
\IfFileExists{upquote.sty}{\usepackage{upquote}}{}
\IfFileExists{microtype.sty}{% use microtype if available
  \usepackage[]{microtype}
  \UseMicrotypeSet[protrusion]{basicmath} % disable protrusion for tt fonts
}{}
\makeatletter
\@ifundefined{KOMAClassName}{% if non-KOMA class
  \IfFileExists{parskip.sty}{%
    \usepackage{parskip}
  }{% else
    \setlength{\parindent}{0pt}
    \setlength{\parskip}{6pt plus 2pt minus 1pt}}
}{% if KOMA class
  \KOMAoptions{parskip=half}}
\makeatother
\usepackage{xcolor}
\IfFileExists{xurl.sty}{\usepackage{xurl}}{} % add URL line breaks if available
\IfFileExists{bookmark.sty}{\usepackage{bookmark}}{\usepackage{hyperref}}
\hypersetup{
  pdftitle={Philosophie - l'état à-t-il tous les droits ?},
  hidelinks,
  pdfcreator={LaTeX via pandoc}}
\urlstyle{same} % disable monospaced font for URLs
\setlength{\emergencystretch}{3em} % prevent overfull lines
\providecommand{\tightlist}{%
  \setlength{\itemsep}{0pt}\setlength{\parskip}{0pt}}
\setcounter{secnumdepth}{-\maxdimen} % remove section numbering

\title{Philosophie - l'état à-t-il tous les droits ?}
\author{}
\date{}

\begin{document}
\maketitle

\hypertarget{probluxe9matique-letat-uxe0-t-il-tous-les-droits}{%
\section{Problématique : L'Etat à-t-il tous les droits
?}\label{probluxe9matique-letat-uxe0-t-il-tous-les-droits}}

On confie le pouv. à l'état pour qu'il assure notre sécurité

Nos droits apparaissent comme des limites au pouvoir de l'Etat

Les droits individuels s'opposent aux droits (pouvoirs) de l'Etat

Donc : pour garantir nos droits, il faut étendre le pouvoir de l'Etat

MAIS si l'Etat possède trop de pouvoirs (droits), ce sont nos droits
individuels qui s'en trouvent menacés

Etat à trop de pouvoir \(\implies\) droits individuels menacés

\hypertarget{i-un-etat-tout-puissant}{%
\section{I) Un Etat tout puissant}\label{i-un-etat-tout-puissant}}

\hypertarget{a-thomas-hobbes}{%
\subsection{A) Thomas Hobbes}\label{a-thomas-hobbes}}

\begin{itemize}
\tightlist
\item
  Philosophe anglais, XVII siècle
\item
  Contemporain de la guerre civile anglaise
\item
  L'auteur du \emph{Léviathan} (son ouvrage de philosophie politique
  majeur)
\end{itemize}

\hypertarget{luxe9tat-de-nature}{%
\subsubsection{1. L'état de nature}\label{luxe9tat-de-nature}}

Une expérience de pensée par laquelle le philosophe cherche à imaginer
ce qu'a pu être la vie humaine \textbf{AVANT} que soit instauré un
``\emph{état civil}'', la vie en société. \(\implies\) l'état de nature
précède et s'oppose à l'état de société / civil

Question de Hobbes : à quoi pouvait bien ressembler la vie humaine avant
la vie en société ? Qu'est-ce que serait la vie humaine hors de l'état
social ?

Situation d'\textbf{anarchie} : pas de pouvoir central

Situation de \textbf{survie} : chacun se débrouille comme il peut pour
\textbf{persister dans l'existence}

Le but de chaque individu isolé c'est d'assurer sa propre conservation,
la conservation de sa propre personne.

Or, dans cet état il n'y a pas d'Etat

On est face à la \textbf{loi du plus fort}

Chacun lutte pour sa survie contre tous les autres: l'état de nature se
caractérise par une guerre \textbf{``de chacun contre chacun''}

Guerre sans limites, pas de limites de \textbf{temps} (tant que l'état
de nature dure, la guerre continue), pas de limites \textbf{d'espace}
(c'est la guerre partout), pas de limites de \textbf{moyens} (tout est
bon pour survivre)

Situation de très grande \textbf{précarité} : La survie de chacun est le
but, mais sa réalisation est très incertaine

L'état de nature est un état de relative égalité entre tous

Tous ont autant de chance de survivre, ou de périr, et aucun avantage
définitif n'est jamais acquis

Rien ne dure, rien n'est stable dans ce monde.

``La vie humaine {[}à l'état de nature{]} est solitaire, misérable,
dangereuse, animale, et brève'' (\emph{Le Léviathan})

\hypertarget{solitaire}{%
\paragraph{Solitaire}\label{solitaire}}

\begin{itemize}
\tightlist
\item
  pcq'il n'y pas de société
\item
  pcq toute confiance envers autrui est impossible : autrui est tjrs
  dans cet état un ennemi en puissance
\end{itemize}

\hypertarget{misuxe8reuse}{%
\paragraph{Misèreuse}\label{misuxe8reuse}}

\begin{itemize}
\tightlist
\item
  impossibilité de stocker des biens : tout bien fait de vous une cible
  potentielle
\item
  Impossibilité de développer des activités économiques au long cours
  (pas d'agriculture, d'élevage)
\end{itemize}

Si ce qui différencie l'homme de l'animal c'est la vie sous \textbf{des
lois communes}, fondée sur des échanges et sur des capacités techniques,
alors la vie à l'état de nature apparait comme quelque chose d'animal.

L'\textbf{état de nature} est une forme de vie peu enviable et
caractérisé par la ``\textbf{loi du plus fort}''.

Si à l'état de nature n'existe \textbf{aucune loi} sauf celle ``du plus
fort'', et s'il n'existe en réalité \textbf{personne qui soit ``le plus
fort''}, alors la notion même de loi ne correspond à rien : il n'y a pas
de lois

L'état de nature est un état \textbf{sans lois} : Rien n'est interdit,
Tout est permis

Si j'ai la force de faire une chose, alors cette chose m'est autorisée.

La \textbf{loi du plus fort} ou la \textbf{loi de la jungle}
\(\implies\) (paradoxe) situations sans \textbf{aucune loi}

\textbf{loi du plus fort : avoir la force de faire = avoir le droit de
faire.}

L'état de nature est un état peu enviable.

\textbf{Le but de l'homme c'est la conservation de sa propre personne}

Or, l'état de nature menace cette conservation : l'homme étant doué de
raison, il voit que cet état de nature n'est pas satisfaisant et il faut
donc en sortir.

\hypertarget{comment-sortir-de-luxe9tat-de-nature}{%
\subsubsection{2. Comment sortir de l'état de nature
?}\label{comment-sortir-de-luxe9tat-de-nature}}

L'homme a l'état de nature jouit d'une \textbf{complète liberté}, il a
tous les droits - tout lui est permis, pourvu qu'il juge que c'est
nécessaire pour sa survie. Rien ne le limite dans ses actions.

Comme l'état de nature, cad l'état d'une \textbf{liberté totale} est
aussi l'état où notre \textbf{survie est la plus menacée}, il faut pour
assurer notre conservation, abandonner (un peu de) notre liberté en vue
de notre sécurité : liberté \(\longmapsto\) sécurité

Chacun va passer un \textbf{PACTE}, ou encore un \textbf{CONTRAT}, avec
chacun des autres pour dire ``pourvu que tu ne m'attaques pas, je ne
t'attaquerai pas non plus''

Chacun promet de se restreindre lui-même, de limiter,
d'\textbf{abandonner un peu de sa liberté, pourvu que les autres fassent
de même.}

Ce pacte ou ce contrat est ce qu'on appelle le \textbf{CONTRAT SOCIAL}

Les théoriciens du contrat social (Hobbes, Locke, Rousseau, etc.) sont
des ``\emph{contractualistes}''. Un contractualiste, c'est quelqu'un qui
théorise l'émergence de la vie sociale à partir d'un pacte ou d'un
contrat passé par les individus à l'état de nature pour sortir de cet
état.

pour que le contrat fonctionne, il faut qu'un \textbf{tiers} assure à
chacun de nous que l'autre respectera sa part du contrat. Pour que ce
tiers puisse garantir que chacun respecte sa part du contrat, il faut
qu'il soit en mesure de \textbf{contraindre chacun} à ce respect s'il
nous venait l'idée de chercher à revenir sur la parole donnée Donc ce
qu'il faut c'est un \textbf{tiers tout puissant} vis-à-vis de chacun de
nous pris séparément. Ce tiers tout puissant à l'égard des individus
c'est ce qu'on appelle \textbf{l'Etat}.

Dans l'acte du contrat :

\begin{enumerate}
\def\labelenumi{\arabic{enumi}.}
\tightlist
\item
  on renonce à son droit universel sur toutes choses, cad, à sa liberté
  totale
\item
  on transfert notre pouvoir sur toutes choses à un tiers, l'Etat, qui
  est chargé de l'exercer en notre nom.
\end{enumerate}

C'est l'idée de Hobbes : il faut une toute puissance de l'Etat. Les
contractualistes ultérieurs remettent cela en question

\hypertarget{letat-un-luxe9viathan}{%
\paragraph{3. L'Etat, un Léviathan}\label{letat-un-luxe9viathan}}

\textbackslash includegraphics{[}scale=0.67{]}\{leviathan.png\} - Épée :
symbole de force

\begin{itemize}
\item
  Sceptre : symbole de \textbf{souveraineté}

  \(\implies\) le souverain est celui \textbf{dont la volonté à valeur
  de loi}
\end{itemize}

\end{document}
